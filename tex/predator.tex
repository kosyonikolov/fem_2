\documentclass[fleqn,12pt]{article}

\usepackage[margin=15mm]{geometry}
\usepackage[utf8]{inputenc}
\usepackage[bulgarian]{babel}
\usepackage[unicode]{hyperref}
\usepackage{amsthm}
\usepackage{amssymb}
\usepackage{mathtools}
\usepackage[unicode]{hyperref}
\usepackage{enumitem, hyperref}
\usepackage{indentfirst}
\usepackage{graphics}

\renewcommand{\arraystretch}{1.3}   

\title{Rosenzweig-McArthur - МКЕ}
\author{Косьо Николов}
\date{май 2023}

\begin{document}
    
\maketitle

\tableofcontents
\pagebreak

\section{Дискретизация по простанството}
Умножаваме скаларно двете уравнение с функция $v \in H_1$:
\[ \left(\frac{\partial N}{\partial t}, v\right) - (\Delta N, v) + (N, v) - \frac{1}{0.4} (N^2, v) - (Q, v) = 0\]
\[ \left(\frac{\partial P}{\partial t}, v\right) - (\Delta P, v) - 0.2(P, v) + (Q, v) = 0\]
\[ Q(N, P) = \frac{0.5 NP}{0.5 + N}\]

Имаме хомогенни условия на Нойман за $N$ и $P \Rightarrow (\Delta N, v) = (\nabla N, \nabla v), (\Delta P, v) = (\nabla P, \nabla v)$. 
Тогава системата ни става
\[ \left(\frac{\partial N}{\partial t}, v\right) - (\nabla N, \nabla v) + (N, v) - \left(\frac{N^2}{0.4}  + Q, v\right) = 0\]
\[ \left(\frac{\partial P}{\partial t}, v\right) - (\nabla P, \nabla v) - 0.2(P, v) + (Q, v) = 0\]

Нека областта $\Omega$ е дискретизирана с мрежата $\mathcal{K}$, на чиито възли съотвестват базисните функции
$\{\varphi_i\}_0^M$. Ще търсим приближени решения във вида
\[ N_h(x,y,t) = \sum_{i=0}^M q_{Ni}(t) \varphi_i(x,y,t) \hspace{10mm}
P_h(x,y,t) = \sum_{i=0}^M q_{Pi}(t) \varphi_i(x,y,t) \]

Съответно ще изискваме вариационните уравнения да са удовлетрворени за всяко $v \in V_h \Leftrightarrow$
да са удовлетрворени за $v = \varphi_j, j = 0, 1, \dots, M$.
Нелинейните членове биха изисквали пресмятане на матрици на всяка стъпка по времето (и итерация на алгоритъма за решаване на нелинйената система),
затова ще ги интерполирима по същия базис. Обозначаваме $Q_1 = \frac{N^2}{0.4} + Q$ и получаваме
\[ Q_h = \sum_{i=0}^M Q(q_{Ni}, q_{Pi}) \varphi_i \hspace{10mm}
Q_{1h} = \sum_{i=0}^M Q_1(q_{Ni}, q_{Pi}) \varphi_i \]

Остава да запишем вариационните уравнения за приближението решение. Базисните функции не зависят от $t$ - можем да
извадим $q$-тата от първия член:
\[ \sum_{i=0}^M \frac{\partial q_{Ni}(t)}{\partial t} (\varphi_i, \varphi_j) - \sum_{i=0}^M q_{Ni}(t) (\nabla \varphi_i, \nabla \varphi_j) + \sum_{i=0}^M q_{Ni}(t) (\varphi_i, \varphi_j) - \sum_{i=0}^M Q_1(q_{Ni}, q_{Pi})(\varphi_i, \varphi_j) = 0\]
\[ \sum_{i=0}^M \frac{\partial q_{Pi}(t)}{\partial t} (\varphi_i, \varphi_j) - \sum_{i=0}^M q_{Pi}(t) (\nabla \varphi_i, \nabla \varphi_j) - 0.2\sum_{i=0}^M q_{Pi}(t) (\varphi_i, \varphi_j) + \sum_{i=0}^M Q(q_{Ni}, q_{Pi})(\varphi_i, \varphi_j) = 0\]

Това трябва да е изпълнено за всяко $j \in [0, \dots, M] \Rightarrow$ можем да запишем системата в матричен вид:
\[ M_0 \frac{\partial \pmb{q}_{N}}{\partial t} - M_1 \pmb{q}_N + M_0 (\pmb{q}_N - \pmb{Q}_1(\pmb{q}_N, \pmb{q}_P)) = 0\]
\[ M_0 \frac{\partial \pmb{q}_{P}}{\partial t} - M_1 \pmb{q}_P + M_0 (-0.2 \pmb{q}_P + \pmb{Q}(\pmb{q}_N, \pmb{q}_P)) = 0\]

Тук $M_0$ и $M_1$ са стандартните матрици на маса и коравина. 

\section{Блочно-матрична форма на полу-дискретното уравнение}
Дефинираме $\pmb{q} = (\pmb{q}_N, \pmb{q}_P)$, т.е.
конкатенецията на двата вектора с възлови неизвестни. Тогава можем да запишем системата в блочна форма:
\[ \begin{pmatrix} M_0 & \pmb{0} \\ \pmb{0} & M_0 \end{pmatrix} \frac{\partial \pmb{q}}{\partial t}
-  \begin{pmatrix} M_1 & \pmb{0} \\ \pmb{0} & M_1 \end{pmatrix} \pmb{q} 
+ \begin{pmatrix} M_0 & \pmb{0} \\ \pmb{0} & M_0 \end{pmatrix} \begin{pmatrix} \pmb{q}_N - \pmb{Q}_1(\pmb{q}_N, \pmb{q}_P) \\ -0.2 \pmb{q}_P + \pmb{Q}(\pmb{q}_N, \pmb{q}_P) \end{pmatrix} = \pmb{0} \]

Въвеждаме допълнителни означения:
\[ M_0^* = \begin{pmatrix} M_0 & \pmb{0} \\ \pmb{0} & M_0 \end{pmatrix} \hspace{10mm}
M_1^* = \begin{pmatrix} M_1 & \pmb{0} \\ \pmb{0} & M_1 \end{pmatrix} \hspace{10mm}
\pmb{r}(\pmb{q}) = \begin{pmatrix} \pmb{q}_N - \pmb{Q}_1(\pmb{q}_N, \pmb{q}_P) \\ -0.2 \pmb{q}_P + \pmb{Q}(\pmb{q}_N, \pmb{q}_P) \end{pmatrix} \]

Така съкратихме системата до 
\[ M_0^* \frac{\partial \pmb{q}}{\partial t} - M_1^* \pmb{q} + M_0^* \pmb{r}(\pmb{q}) = \pmb{0}  \]

\section{Неявен метод на Ойлер}
Нека имаме два слоя по времето - $\pmb{q}_{i}$ и $\pmb{q}_{i+1}$ - разделени със стъпка $\tau$.
Неявният метод на Ойлер изисква всичко в "дясната страна" (различно от производната по времето)
да се оцени на $i+1$-вия слой:
\[ M_0^* \frac{\pmb{q}_{i+1} - \pmb{q}_i}{\tau} - M_1^* \pmb{q}_{i+1} + M_0^* \pmb{r}(\pmb{q}_{i+1}) = \pmb{0}  
\Leftrightarrow M_0^* (\pmb{q}_{i+1} - \pmb{q}_i) - \tau M_1^* \pmb{q}_{i+1} + \tau M_0^* \pmb{r}(\pmb{q}_{i+1}) = \pmb{0} 
\Leftrightarrow\]
\[ \Leftrightarrow (M_0^* - \tau M_1^*) \pmb{q}_{i+1} + \tau M_0^* \pmb{r}(\pmb{q}_{i+1}) - M_0^* \pmb{q}_i = \pmb{0} \]

Системата е нелинейна заради реакционния член $\pmb{r}$ и ще трябва да се ползва итеративен метод за решаване на всеки слой по времето.
Хубавото е, че матриците са константни и трябва да ги сметнем само веднъж. За пълнота, нека запишем и явният вид на $\pmb{r}$:
\[ \pmb{r}(\pmb{q}) = \begin{pmatrix} 
    \pmb{q}_N - \dfrac{0.5 \pmb{q}_N \cdot \pmb{q}_P}{0.5 + \pmb{q}_N} - \dfrac{\pmb{q}_N^2}{0.4} \\[5mm]
    -0.2 \pmb{q}_P + \dfrac{0.5 \pmb{q}_N \cdot \pmb{q}_P}{0.5 + \pmb{q}_N} 
\end{pmatrix} \]
Умножението, делението и вдигането на степен на векторите тук са поелементни.

\end{document}